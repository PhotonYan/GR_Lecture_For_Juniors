\documentclass{article}
\usepackage{ctex}
\usepackage{bm}
\usepackage{tikz}
\usepackage{amsmath}
\usepackage{graphicx}

\begin{document}
\title{写给初中生的广义相对论入门}
\author{近代物理社}
\maketitle
我们重新定义一种时间和距离的单位使得这一单位制下光速$c=1$。
\section{时空的几何学}
我们每日身处时空当中,你是否曾追问过:时空是什么?我们用物理学家的观念对这一哲学问题作以阐释,我们愿意将时空抽象为包含我们认为时空本应该包含的信息的一个数学对象。

首先,我们世界里某一个特定时刻的特定位置,都应该是时空的一个组成部分,我们管这一个“点”的概念称为事件,我们说:时空是所有事件的集合。

这个集合里有许多直观的结构:比如在一个不是很大的房间里,在不长的一段时间中,可以用一个钟定义房间里的时间$t$,并在房间里画上三维坐标$O-xyz$,那么给定坐标$(x^0,x^1,x^2,x^3)=(t,x,y,z)$,我们便可以确定这段时间房间中的一个事件。

而且,我们有很直观的感受说如果两个事件发生时间临近位置也临近,这两个事件是邻近的。因此如果给我们一张四维的纸,我们可以把那片时空的地图画在纸上,使临近的事件对应的点也临近,并且地图上有着坐标。

如果我们对一片大时空中可以分割出来的每一个奇形怪状的部分都画上地图并粘在一起,就得到了整个时空的大地图。如果把我们所说的上述结构的构造数学化,我们可以说:时空是一个微分流形。

但是凡是坐飞机时看过地图上飞机的航线的同学都知道,地球球面上连接地球上两点的最短线在地图上看是曲线。这是因为地图上的距离是失真的。

我们想描述时空的几何结构不能只有地图这一张图(初中学地理大家知道地图一定要有比例尺),我们必须明确时空的地图上的“距离(长度)”概念。

必须注意的是,这个距离的概念在地图上不同的点是不同的,因为一般而言时空的几何是弯曲的,类似于地球那样。因此,我们必须逐点给出我们的距离失真才可以真正标定距离概念。

我们还要注意,有限间隔的两点间一般而言距离是曲线长度,不方便定义。

因为当你看一个弯曲的微分流形某点附近很小的范围,它近似是平直的,我们希望定义一点附近非常临近的两点连线作为矢量,并定义这个矢量的模长平方在地图上的算法。

数学上严格构造微分流形上的矢量的方式如下:

由初中几何中的勾股定理
\end{document}