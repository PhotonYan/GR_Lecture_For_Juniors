\documentclass[12pt,a4paper,violet]{ctexart}
\usepackage{ctex}
\usepackage{amsmath}
\usepackage{bm}
\usepackage{fontspec}
\usepackage{blindtext}
\usepackage{wrapfig}
\usepackage{CJKnumb}
\usepackage{indentfirst}
\usepackage{subfiles}
%\usetikzlibrary{decorations.markings,arrows}
\setlength{\parindent}{3em}
\usepackage{color}

\usepackage{mathrsfs}
\usepackage{longtable}
\usepackage{booktabs}

%\newtheorem{extend}{[延伸阅读]}
\renewcommand{\d}{\mathrm{d}}
\renewcommand{\i}{\text{i}}
\newcommand{\e}{\text{e}}
\renewcommand{\#}{\tilde}

\begin{document}

\LaTeX


\end{document}
\subsection{正文之前}
在本书中将有以下几种功能框,我们先来认识一下吧!
\begin{definition}
    这是一个定义框。
\end{definition}
\begin{theorem}
    这是一个定理框。
\end{theorem}
\begin{exercise}
    这是一个练习题框,一般来说会有答案,以“答案”框的形式出现在本框下方。如果没有答案,则请读者自行找到答案。
\end{exercise} 
\begin{answer}
    一般作为“练习”的答案。
\end{answer}
\begin{example}
    这是例题框。例题的题干将写在这里。
\end{example}
\begin{solution}
    这是例题的解答。
\end{solution}
\begin{Formula}
    这是一个公式框,内部会有一个重要公式,例如:
    \begin{equation}
        1+1=2
    \end{equation}
\end{Formula}
\begin{remark}
    这是提示框,主要写的是有关做题技巧的描述或旁批内容。
\end{remark}
\[\begin{aligned}

\end{aligned}\]